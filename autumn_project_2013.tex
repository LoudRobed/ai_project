% (C) Anders Kofod-Petersen
\documentclass[a4paper]{book}
\usepackage[english]{babel}						% Correct English hyphenation
\usepackage[latin1]{inputenc}						% Allow for non-English letters
\usepackage{graphicx}							% To include graphics
\usepackage{natbib}								% Correct citations
%\usepackage{fancyheadings}						% Nice header and footer
\usepackage[linktocpage,colorlinks]{hyperref}			% PDF hyperlink
\usepackage{geometry} 							% Better geometry
\usepackage{mathtools}                          % for typing math
\usepackage{amsmath}
%\usepackage[center]					% For cropping documents

% B5 (uncomment to convert to B5 format)
 \geometry{b5paper}

% Author
% Fill in here, and use commands in the text. 
\newcommand{\thesisAuthor}{Andreas Hagen}
\newcommand{\thesisTitle}{Evolving Artificial Altruism}
\newcommand{\thesisType}{Specialisation project}
\newcommand{\thesisDate}{fall 2013}

% PDF info
\hypersetup{pdfauthor={\thesisAuthor}}
\hypersetup{pdftitle={\thesisTitle}}
\hypersetup{pdfsubject={\thesisType}}
\hypersetup{linkcolor=black}
\hypersetup{citecolor=black}
\hypersetup{urlcolor=black}

%Fancy headings
%\pagestyle{fancy}
%\pagestyle{fancyplain}
%\renewcommand{\chaptermark}[1]{\markboth{#1}{}}
%\renewcommand{\sectionmark}[1]{\markright{#1}{}}
%\lhead[\fancyplain{}{\thepage}]{\fancyplain{}{\let\uppercase\relax\leftmark}}
%\rhead[\fancyplain{}{\let\uppercase\relax\rightmark}]{\fancyplain{}{\thepage}}
%\chead[\fancyplain{}{}]{\fancyplain{}{}}
%\lfoot[\fancyplain{}{}]{\fancyplain{}{}}
%\cfoot[\fancyplain{}{}]{\fancyplain{}{}}
%\rfoot[\fancyplain{}{}]{\fancyplain{}{}}

% Citation format
\bibliographystyle{apalike}
\bibpunct{[}{]}{;}{a}{,}{,}

\begin{document}

%Title page (This is generate automatically from the commands above)
\begin{titlepage}
\noindent {\large \textbf{\thesisAuthor}}
\vspace{2cm}

\noindent {\Huge \thesisTitle}
\vspace{2cm}

\noindent \thesisType, \thesisDate 
\vspace{2cm}

\noindent Artificial Intelligence Group\\ Department of Computer and Information Science\\ Faculty of Information Technology, Mathematics and Electrical Engineering\\

\vfill
\begin{center}
\includegraphics[width=3cm]{ntnu.pdf}
\end{center}
\end{titlepage}

\thispagestyle{empty}

\cleardoublepage

\frontmatter

\section*{Abstract}


In this paper the subject of exploring ways of evolving altruistic behaviour in distributed autonomous robots is explored.
The research is motivated by the need for autonomous evolving robots to evolve relevant traits needed to avoid extinction in a dynamic environment. 
The paper contains a structured litterature review on the field of altruism in artificial evolution that places this work in context with contemporary research. 

%\begin{itemize}
%\item the field of research
%\item a brief motivation for the work
%\item what the research topic is and
%\item the research approach(es) applied. 
%\item contributions
%\end{itemize}

%The abstract length should be roughly half a page of text --- without lists, tables or figures.  

\clearpage

\section*{Preface}



\vspace{1cm}

This paper was written as a specialisation project in the autumn of 2013 at the Norwegian University of Science and Technology. It was officially supervised by Pauline Haddow with assistance of Jean-Marc Montanier.
%The preface includes the facts - what type of project, where it is conducted, who supervised and any acknowledgements you wish to give. 

\vfill

\hfill \thesisAuthor

\hfill Trondheim, \today

\clearpage

\tableofcontents

%\listoffigures

%\listoftables

\mainmatter

\chapter{Introduction}
\label{cha:Introduction}

In recent years the new field of swarm robotics has emerged from the field of bio-inspired artificial intelligence. The concept involves a distributed set of autonomus robots that adhere to a set of rules creating simple behaviours. The goal of this is to have the group of robots perform small sub tasks that alone are simple, but that put together achieves complex goals. 
One way of creating control mechanisms for such robots is to use artificial neural networks
 minimaly

%All chapters should begin with an introduction before any sections begin. Further, each sections begins with an introduction before  subsections begin. Chapters with just one section or sections with just one sub-section, should be avoided. Think carefully about chapter and section titles as each title stand alone in the table of contents (without associated text) and should convey meaning for the contents of the chapter or section. 

%In all chapters and sections it is important to write clearly and concisely. Avoid repetitions and if needed, refer back to the original discussion or presentation. Each new section, subsection or paragraph should provide the reader with new information and be written in your own words. Avoid direct quotes. If you use direct quotes, unless the quote itself is very significant, you are conveying to the reader that you are unable to express this discussion or fact yourself. Such direct quotes also break the flow of the language (yours to someone else's).   





\section{Background and Motivation}\label{cit}
\label{sec:BackgroundAndMotivation}




%Having a template to work from provides a starting point. However, for a given project, a slight variation in the template may be required due to the nature of the given project. Further, the order in which the various chapters and sections will be written will also vary from project to project but will seldom start at the abstract and sequentially follow the chapters of the report. One critical reason for this, is that you need to start writing as early as possible and you will begin to write up where you are currently focusing. However, do not leave the abstract until the end. The abstract is the first thing anyone reads of an article or thesis --- after the title; and thus it is important that it is very well written. Abstracts are hard to write so create revisions throughout the course of your project as your project progresses.  

%This introduction to background and motivation should state where this project is situated in the field and what the key driving forces motivating this research are. However, keep this section brief as it is still part of the introduction. The motivation will be further extended in chapter~\ref{T-B}, presenting your complete state-of-the-art. 

%Note that this template uses italics to highlight where latin wording is inserted to represent text and the text of the template that we wish to draw your attention to. The italics themself are not an indication that such sections should use italics.  

\section{Goals and Research Questions}
\label{sec:Goals and Research Questions}

%A masters is a research project and thus there needs to be a question(s) that need answered. Such questions are often a very important part of the results that come out of the specialisation project. For those following the one year masters project, it is desirable to create such questions as early as possible as   The formation of such questions provide both an important driving force for the masters project and provide clarity as to the goals sought. However, one will expect to refine the questions and thus the final path of the masters as work progresses. However any refinements should be conducted with care so as to avoid that the original aims, and previous work are not lost.  
%It is always good to have one (or max 2) key questions and perhaps some sub questions. 

\begin{description}
\item[Goal] {\it Evolve observable and quantifiable altruistic beahaviour using open-ended evolution, ie. without specialiced global fitness functions.}
\end{description}

Natural evolution can be seen as a response to the environment, and it is therefore desirable to develop genetic algorithms that are capable of evolving new behaviours in response to a dynamic environment. This entails evolving solutions to problems that are not known in advance and this is the rationale behind wanting to evolve advanced behaviour without specifically choosing genotypes based on the traits we want to evolve. 

The grander goal here is to work towards a robust and agile evolutionary algorithm for distributed autonomous robots that can adapt to unexpected change.

%Your goal/objective should be described in a single sentence. In the text under you can expand on this sentence to clarify what is meant by the short goal description. 
%The goal of your work is what you are trying to achieve. This can either be the goal of your actual project or can be a broader goal that you have taken steps towards achieving. Such steps should be expressed in the research questions. 
%Note that the goal is seldom to build a system. A system is built to to enable experiments to be conducted. The research question/goal would be the goal that the system is implemented to meet.  


\begin{description}
\item[Research question 1] {\it What selection methods are needed to elicit an altruistic response without specifically rewarding it?}
\end{description}

Some more explanation here

%Each research question provides a sub-goal and these should be precise and clearly stated enabling the reader to match your results to the original goals. They will also form the driving force for the experimental plan. 

%\begin{description}
%\item[Research question 2] {\it Lorem ipsum dolor sit amet, consectetur adipiscing elit.}
%\end{description}

%{\it Lorem ipsum dolor sit amet, consectetur adipiscing elit. Nam consequat pulvinar hendrerit. Praesent sit amet elementum ipsum. Praesent id suscipit est. Maecenas gravida pretium magna non }

%\section{Research Method}
%\label{sec:researchMethod}

%What methodology will you apply to address the goals: theoretic/analytic, model/abstraction or design/experiment? This section will describe the research methodology applied and the reason for this choice of research methodology.  

%\section{Contributions}
%\label{sec:IntroContributions}
%
%The main description of the contributions will come in chapter~\ref{cont} after the results are presented. This section just provides a brief summary of the main contributions of the work. This section can also be left out, leaving all discussions in chapter~\ref{cont}.
%
%The format of this section will generally follow the following format:
%{\it
%Donec non turpis nec neque egestas faucibus nec id neque. Etiam consectetur, odio vitae gravida tempus, diam velit sagittis turpis, a molestie ligula tellus at nunc. Nam convallis consequat vestibulum. Proin dolor neque, dapibus a pellentesque a, commodo a nibh.}

%\begin{enumerate}
%\item {\it Lorem ipsum dolor sit amet, consectetur adipiscing elit.}
%\item {\it Lorem ipsum dolor sit amet, consectetur adipiscing elit.}
%\item {\it Lorem ipsum dolor sit amet, consectetur adipiscing elit.}
%\end{enumerate}
%

%\section{Thesis Structure}
%\label{sec:thesisStructure}

%This section provides the reader with an overview of what is coming in the next chapters. You want to say more than what is explicit in the chapter name, if possible, but still keep the description short and to the point. 


%\chapter{Background Theory and Motivation}\label{T-B}
%\label{cha:TheoryAndBackground}

%{\it Lorem ipsum dolor sit amet, consectetur adipiscing elit. Nam consequat pulvinar hendrerit. Praesent sit amet elementum ipsum. Praesent id suscipit est. Maecenas gravida pretium magna non interdum. Donec augue felis, rhoncus quis laoreet sed, gravida nec nisi. Fusce iaculis fermentum elit in suscipit.}


%\section{Background Theory}
%\label{sec:no1}

%The background theory depth and breadth depends on the depth needed to understand your project in the different disciplines that your project crosses.  It is not a place to just write about everything you know that is vaguely connected to your project. The theory is here to help the reader that does not know the theoretical basis of your work so that he/she can gain sufficient understanding to understand your contributions. In particular, the theory section provides an opportunity to introduce terminology that can later be used without disturbing the text with a definition.  In some cases it will be more appropriate to have a separate section for different theory. However, watch that you don't end up with too short sections. Subsections may also be used to separate different background theory. 

%When introducing techniques or results, always reference the source. Be careful to reference the original contributor of a technique and not just someone who happens to use the technique. For relevant results to your work, you would want to look particularly at newer results so that you have referenced the most up-to-date work in your area. If you don't have the source handy when writing, mark the test that a reference is needed and add it later. 

%Web pages are not reliable sources --- they might be there one day and removed the next; and thus should be avoided, if possible. A verbal discussion is not a source and should not be referenced or described in the text.  

%The bulk of citations in the report will appear in section~\ref{cit}. However, you will often need to introduce some terminology and key citations already in this chapter. 

%You can cite a paper in the following manners: 

%\begin{itemize}
%\item when referring to authors:\\
 %\citet{authorson10:_secon_best_paper_in_world} stated something rather nice.
%\item to cite indirectly: \\
% Papers should be written nicely \citep{authorson10:_secon_best_paper_in_world}\\
%or\\
%In \cite{authorson10:_secon_best_paper_in_world}, a less detailed template was presented.
%\item To just cite the authors: \\
%\citeauthor{authorson10:_secon_best_paper_in_world} wrote a nice paper.
%\item Or just the year: \citeyear{authorson10:_secon_best_paper_in_world}.
%\item You can even cite specific pages: \citet[p. 3]{authorson10:_secon_best_paper_in_world}.
%\end{itemize}
%
%\vspace{0.5cm}
%
%\noindent
%{\bf Introducing figures:} \\

%\begin{figure}[ht]
%\begin{center}
%\includegraphics[width=0.5\columnwidth]{figs/figure1.pdf}
%\caption[Boxes and arrows are nice]{Boxes and arrows are nice (adapted from \citet{authorson10:_secon_best_paper_in_world})}
%\label{fig:BoxesAndArrowsAreNice}
%\end{center}
%\end{figure}

%Remember that when you borrow figures you should always credit the original author --- such as Figure \ref{fig:BoxesAndArrowsAreNice} (adapted from \citet{authorson10:_secon_best_paper_in_world}). Also don't just put the figure in and leave it to the author to try to understand what the figure is. The figure should be put in to convey a message and you need to help the author to understand the message intended by explaining the figure in the text. 
%
%\vspace{0.5cm}
%
%\noindent
%{\bf Introducing tables in the report: }\\

%\begin{table}[htdp]
%\begin{center}
%\begin{tabular}{|c|c|c|c|c|}\hline\hline
%This & is & a & nice & table\\\hline
%This & is & a & nice & table\\\hline\hline
%\end{tabular}
%\caption{Example Table}
%\end{center}
%\label{tab:ExampleTable}
%\end{table}%

%As you can see from Table \ref{tab:ExampleTable}, tables are nice. However, again, you need to discuss the contents of the table in the text. You don't need to describe every entry but draw the authors attention to what is important for he/she to glean from the table. 


\chapter{Evolving Altruism: A Systematic Literature Review}\label{T-B}
\label{cha:STL}

\section{Introduction}
\label{sec:STLintro}

The systematic literature review was performed using the guidelines for systematic 
literature review in software engineering presented in \cite{07guidelinesfor}
The literature review process is divided into 8 steps: %Check Kitchenham 2007

\begin{description}
\item[Step 1] {\it Defining review questions}

\item[Step 2] {\it Defining the systematic literature review protocol}

\item[Step 3] {\it Search for relevant studies}

\item[Step 4] {\it Selection of studies}

\item[Step 5] {\it Quality assessment}

\item[Step 6] {\it Data Collection}

\item[Step 7] {\it Data synthesis and anaylysis}

\item[Step8] {\it Dissemination}

\end{description}

\subsection{Defining the review questions}
The first step in the systematic review process was to formalize the the goal of the review into review questions that the review is meant to answer. The goal of the review was to answer the following questions:

\begin{description}
\item[RQ1] {\it What kind of research has already been concluded that focuses on the artificial evolution of altruistic behaviour?}

\item[RQ2] {\it What are the conditions believed to be needed in nature for the evolution of altruistic behaviour to occur in nature?}

\item[RQ3] {\it What exists in terms of evolutionary models that reward only survival in the environment instead of rewarding specific types of behaviour?}

\end{description}

\subsection{Defining the Systematic review Protocol}

I'm not sure what to write here as I'm detailing the protocol as I go along.

\subsection{Search for relevant studies}

To perform the search in a systematic way I compiled a list of of relevant sources which would be the subject to systematic query. I decided to use the list compiled in \cite{Lillegraven354464} as a starting point as it presented a list of relevant sources both for research on computer science in general and Artificial Intelligence. In Addition to the sources identified there I also added Google Scholar and Science Direct as suggested by 
\cite{Brereton2007571}. Google Scholar would be my primary source in identifying relevant papers The list of sources is shown in table 
 


\begin{center}
    \begin{tabular}{| l | c | r| }
    \hline
    Source                  &   Type                        & URL \\ \hline \hline
    ACM Digital Library     &   Digital Library             & \url{http://portal.acm.org/dl.cfm} \\  \hline %Subscription
    IEEE Xplore             &   Digital Library             & \url{http://ieeexplore.ieee.org/} \\ \hline   %Subscription
    CiteSeerX               &   Digital Library             & \url{citeseerx.ist.psu.edu} \\ \hline         %Free
    Web of Knowledge        &   Digital Library             & \url{http://wokinfo.com/} \\ \hline           %Free
    Google Scholar          &   Bibliographic Database      & \url{http://scholar.google.com} \\ \hline     %Free
    Journal of AI Resarch   &   Journal                     & \url{http://jair.org/} \\        \hline             %Free
%    References in papers    &   N/A                         & N/A \\
    \hline
    \label{table:sources}
    \end{tabular}
\end{center}

\subsection{Searching the online resources 1.st pass}

Following the methodology in \cite{Lillegraven354464} I created groups of search terms that were synonyms or similar in meaning. The purpose of this was to exploit the possibility of using boolean search strings in modern digital libraries.

\begin{center}
    \begin{tabular}{| l | c | c | r |}
      \hline
       & Group 1 & Group 2 & Group 3 \\ \hline
       Term 1 & Altruism & Evolution & Swarm Robots \\ \hline
       Term 2 & Tragedy of Commons & Natural Selection & Swarm Intelligence \\ \hline
       Term 3 & Collaboration & Evolving & Swarm Behaviour \\ \hline
       Term 4 & & Evolutionary & \\ \hline
       Term 5 & & Genetic & \\ \hline
       \hline
       \end{tabular}
      \end{center}
      
This provided me with a set of 

\begin{equation}
\label{eq:comb}
3*5*3 = 60
\end{equation}

\noindent
pairs of combinations of search terms that could be combined using the logical
or-operator available on most search engines. I wrote a small java-program (Irrelevant?) for 
combining the different search terms into pairs which created the search string
\\
\noindent
\begin{multline}
\label{eq:search}
("Swarm Behaviour" \lor "Swarm Robotics" \lor "Swarm Intelligence") \land \\
(Evolutionary \lor Evolution \lor Evolving \lor Natural Selection \lor Genetic) \land \\
(Altruism \lor Altruistic \lor Tragedy of Commons) \\
\end{multline}

To limit the number of results, the search was set to only return results published
in the last 5 years. The rationale was that this field of research is still in
its infancy and that the most relevant studies would be the most recent. 

\subsubsection{ACM Digital Library}
For the ACM Digital Library, the number of results on the original search query
was so large that it had to be further limited by only including entries from 
relevant publications. Of the publications that returned matches for the query,
only two of them were included in the final search:

\begin{itemize}

\item{Proceedings of the 13th annual conference companion on Genetic and evolutionary computation (255)}
\item{Proceedings of the 11th International Conference on Autonomous Agents and Multiagent Systems - Volume 3 (175) }
\item{The 10th International Conference on Autonomous Agents and Multiagent Systems - Volume 3 (171)}

\end{itemize}

The search was further refined by adding the constraint that the title had to include

$Altruism \lor Altruistic \lor Tragedy of Commons$

These measures brought the total number of hits down from > 130 000 to a more  
managable 89. 

\subsubsection{Search Results}

Applying the search string in \ref{eq:search} to the sources in \ref{table:sources} yielded the results shown in table \ref{table:SearchResults}

\begin{center}
    \begin{tabular}{|l|r|}
    \hline
    Source & Hits \\ \hline
    Google Scholar & 217 \\ \hline
    CiteSeer    &   25 \\ \hline
    ACM Digital Library & 89 \\ \hline
    IEEE Xplore & 0  \\ \hline
    Web of Knowledge & 0 \\ \hline
    Journal of AI Research & 0 \\ \hline
    References  &   0 \\ 
    \hline
     \label{table:SearchResults}
   \end{tabular}
\end{center}

In addition to this systematic query focusing on the intersection of the relevant terms I also conducted some 
"freehand"-searches that focused on the keywords "Altruism" and "Evolution" focusing only on the results deemed
most relevant by the search enginge, and limiting myself to the first 40-50 results. Qualitatively this yielded 
much faster results than the systematic approach.

\subsection{Searching the online resources 2.nd pass}

After completing the first iteration of online searches I was advised to repeat the search without focusing on swarm robotics
reasoning that the swarm is simply a method in which the evolution is demonstrated rather than a goal in itself. I also removed
the term "Tragedy of Commons"

\begin{center}
    \begin{tabular}{| l | c | r |}
      \hline
       & Group 1 & Group 2 \\ \hline
       Term 1 & Altruism 	& Evolution  \\ \hline
       Term 2 & Self-Sacrifice	& Natural Selection \\ \hline
       Term 3 & 		& Evolving   \\ \hline
       Term 4 & 		& Evolutionary  \\ \hline
       \hline
       \end{tabular}
      \end{center}
      
This provided me with a much smaller set of 

\begin{equation}
\label{eq:comb}
2*4 = 8 
\end{equation}

\noindent
pairs of combinations of search terms that could be combined using the 
boolean operators available on most search engines.

\begin{multline}
\label{eq:search}
(Altruism \lor Altruistic ) \land \\
(Evolution \lor Natural Selection\ \lor Evolving \lor Evolutionary)
\end{multline}
 
\subsection{Searching offline}

In addition to exploring the vast online resources I also searched available literature in the University Library and checked reference lists 
in the articles I read that were of particular interest %why?

\begin{tabular}{| p{5cm} | p{5cm} |}
	\hline
	Title of paper found & Referenced in \\ \hline
        The evolution of cooperation and altruism  a general framework and a classification of models & Evolution of Altruistic Robots \\ \hline
\end{tabular}

	
\subsection{Selection of studies}

After applying the search strategy I began selecting the studies that were relevant for my research questions. To filter the number of studies found I employed a three stage screening process where the set of found articles were gradually culled according to a set of inclusion critereas. The three stage process was:

\begin{itemize}

\item screning based on title
\item Screening based on contents in the Abstract
\item Screening based on full-text reading
\item Screening based on quality

\end{itemize}

\subsubsection{Screening based on title}
Applying the search string on Google Scholar yielded 217 results. Many of these could be excluded based on the title using the exclusion criteria

\begin{description}
\item[EQ1] {\it The main focus of the title is not within the field of computer science}
\item[EQ2] {\it It can be quickly determined from the title that the focus of the research is neither AI nor theoretical biology related to altruism}
\end{description}


\subsubsection{Abstract inclusion criteria screening}

The inclusion criterias that were used for the screening based on the contents in the abstract were:

\begin{description}
\item[IC1] {\it The paper focuses mainly on evolving altruistic behaviour using artificial evolution}
\item[IC2] {\it The paper focuses mainly on the evolution of altruistic behaviour in nature}
\item[IC3] {\it The paper focuses mainly on open-ended evolution in swarm robotics}
\end{description}

\subsubsection{Full text inclusion criteria screening}

\begin{description}
\item[IC4] {\it The paper focuses mainly on evolving altruistic behaviour using artificial evolution}
%\item[IC5] {\it The paper focuses mainly on artificial evolution using global fitness functions}
\item[IC6] {\it The paper studies the environment in which altruistic behaviour occur in nature}
\item[IC7] {\it The paper studies the genetic preconditions for the evolution of altruistic behaviour in nature}
\end{description}

\subsubsection{Full text quality criteria screening}

\begin{description}
\item[QC1] {\it There is a clear statement of the aim of the research} 
\item[QC2] {\it The Study is put into context of other studies and research}
\end{description}

\subsection{Quality assessment}
The papers were screened based on the following quality criteria:





\subsection{List of included papers}

\noindent
\begin{center}
    \begin{tabular}{| p{5cm} | p{3cm} | r |}
      \hline
       Title & Author(s) & Source \\ \hline
       	Evolution of Altruistic Robots & \begin{tabular}{l}Dario Floreano \\ Sara Mitri1\\ Andres Perez-Uribe \\ and Laurent Keller \\ \end{tabular} & Google Scholar  \\ \hline
	The Evolution of Non-reciprocal Altruism & \begin{tabular}{l}Martijn Brinkers \\ Paul den Dulk\end{tabular} & ACM Digital Library \\ \hline
	Evolution of Altruism in Viscous Populations: Effects of Altruism on the Evolution of Migrating Behavior & \begin{tabular}{l}Martijn Brinkers \\ Paul den Dulk\end{tabular} & ACM Digital Library \\ \hline
       \end{tabular}
      \end{center}


\subsection{Data collection}
Given the exploratory nature of this literature review the data collection consisted of (somehow write why I'm listing the results as this)



\subsection{Data synthesis and analysis}
placeholder
\subsection{Dissemination}
placeholder
\section{Structured Literature Review Protocol}
\label{sec:protocol}

Here you need to include your structured review protocol including search engine, search words, research questions  (for search, not the masters research questions), inclusion createrias and evaluation Criterias. 

\section{Motivation}
\label{sec:no2}

%Your motivation can be either application driven or technique/methodology driven. However in both cases, there will be an element of methodology driven due to the research focus of our group and the nature of a masters project.  
%What other research has been conducted in this area and how is it related to your work? The text should clearly illustrate why your goals and research questions are important to address. This section is thus where your literate review will be presented. It is important when presenting the review that you present an overview of the motivating elements of the work going on in your field and how these relate to your proposal, rather than a list of contributors and what they have done. This means that you need to extract the key important factors for your work and discuss how others have addressed each of these factors and what the advantages/disadvantages are with such approaches. As you mention other authors, you should reference their work. Note that the reference list reflects the literature you have read and have cited. This will only be a subset of the literature that you have read.

%\chapter{Architecture/Model}
%\label{sec:architectureAndModel}
%
%Here you will present the architecture or model that you have chosen and that is (or will be) implemented in your work. Note that putting algorithms in your report is not desirable but in certain cases these might be placed in the appendix. Code further be avoided in the report itself but may be delivered in the fashion requested by the supervisor or, in the case of masters delivery, submitted as additional documents. 

%\chapter{Experiments and Results}
%\label{cha:ResearchAndResults}

%{\it Lorem ipsum dolor sit amet, consectetur adipiscing elit. Nam consequat pulvinar hendrerit. Praesent sit amet elementum ipsum. Praesent id suscipit est. Maecenas gravida pretium magna non interdum. Donec augue felis, rhoncus quis laoreet sed, gravida nec nisi. Fusce iaculis fermentum elit in suscipit. Donec rutrum tincidunt tellus, ac tempor diam posuere quis. }
%\section{Experimental Plan}
%\label{sec:experimentalPlan}
%
%Trying and failing is a major part of research. However, to have a chance of success you need a plan driving the experimental research, just as you need a plan for your literature search. Further, plans are made to be revised and this revision ensures that any further decisions made are in line with the work already completed.  

%The plan should include what experiments or series of experiments are planned and what question the individual or set of experiments aim to answer. Such questions should be connected to your research questions so that in the evaluation of your results you can discuss the results wrt to the research questions.  

%\section{Experimental Setup}
%\label{sec:experimentalSetup}
%
%The experimental setup should include all data - parameters etc, that would allow a person to repeat your experiments. 

%\section{Experimental Results}
%\label{sec:experimentalResults}

%Results should be clearly displayed and should provide a suitable representation of your results for the points you wish to make. Graphs should be labeled in a legible font and if more than one result is displayed on the same graph then these should be clearly marked.   Please choose carefully rather than presenting every results. Too much information is hard to read and often hides the key information you wish to present. Make use of statistical methods when presenting results, where possible to strengthen the results.  Further, the format of the presentation of results should be chosen based on what issues in the results you wish to highlight. You may wish to present a subset in the experimental section and provide additional results in the appendix.

%\chapter{Evaluation and Conclusion}
%\label{cha:evaluationAndConclusion}

%{\it Lorem ipsum dolor sit amet, consectetur adipiscing elit. Nam consequat pulvinar hendrerit. Praesent sit amet elementum ipsum. Praesent id suscipit est. Maecenas gravida pretium magna non interdum. Donec augue felis, rhoncus quis laoreet sed, gravida nec nisi. Fusce iaculis fermentum elit in suscipit. }

%\section{Evaluation}
%\label{sec:Evaluation}

%When evaluating your results, avoid drawing grand conclusions, beyond that which your results can infact support. Further, although you may have designed your experiments to answer certain questions, the results may raise other questions in the eyes of the reader. It is important that you study the graphs/tables to look for unusual features/entries and discuss these aswell as discussing the main findings in the results. 

%\section{Discussion}
%\label{sec:Discussion}

%In the discussion it is important to include a discussion of not just the merits of the work conducted but also the limitations. 

%\section{Contributions}~\label{cont}
%\label{sec:Contributions}

%What are the main contributions made to the field and how significant are these contribution.  

%\section{Future Work}
%\label{sec:futureWork}

%Consider where you would like to extend this work. These extensions might either be continuing the ongoing direction or taking a side direction that became obvious during the work. Further, possible solutions to limitations in the work conducted, highlighted in ~\ref{sec:Discussion} may be presented. 

\backmatter

\addcontentsline{toc}{chapter}{Bibliography}
\bibliography{bibliography}

%\chapter{Appendices}
%\label{cha:appendices}

%Personal notes:
%In my review it might be reasonable to create a taxonomy of the research that has already been concluded. (Ie. group different studies together).
%What is the purpose of my review if I don't have any research questions? The purpose must be to consolidate information sources.

\end{document}
