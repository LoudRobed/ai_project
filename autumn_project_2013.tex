% (C) Anders Kofod-Petersen
\documentclass[a4paper]{book}
\usepackage[english]{babel}						% Correct English hyphenation
%\usepackage[latin1]{inputenc}						% Allow for non-English letters
\usepackage[utf8]{inputenc}
\usepackage{graphicx}							% To include graphics
\usepackage{natbib}								% Correct citations
%\usepackage{fancyheadings}						% Nice header and footer
\usepackage[linktocpage,colorlinks]{hyperref}			% PDF hyperlink
\usepackage{geometry} 							% Better geometry
\usepackage{mathtools}                          % for typing math
\usepackage{amsmath}
%\usepackage[center]					% For cropping documents

% B5 (uncomment to convert to B5 format)
 \geometry{b5paper}

% Author
% Fill in here, and use commands in the text. 
\newcommand{\thesisAuthor}{Andreas Hagen}
\newcommand{\thesisTitle}{Evolving Self Sacrifice}
\newcommand{\thesisType}{Specialisation project}
\newcommand{\thesisDate}{fall 2013}

% PDF info
\hypersetup{pdfauthor={\thesisAuthor}}
\hypersetup{pdftitle={\thesisTitle}}
\hypersetup{pdfsubject={\thesisType}}
\hypersetup{linkcolor=black}
\hypersetup{citecolor=black}
\hypersetup{urlcolor=black}

%Fancy headings
%\pagestyle{fancy}
%\pagestyle{fancyplain}
%\renewcommand{\chaptermark}[1]{\markboth{#1}{}}
%\renewcommand{\sectionmark}[1]{\markright{#1}{}}
%\lhead[\fancyplain{}{\thepage}]{\fancyplain{}{\let\uppercase\relax\leftmark}}
%\rhead[\fancyplain{}{\let\uppercase\relax\rightmark}]{\fancyplain{}{\thepage}}
%\chead[\fancyplain{}{}]{\fancyplain{}{}}
%\lfoot[\fancyplain{}{}]{\fancyplain{}{}}
%\cfoot[\fancyplain{}{}]{\fancyplain{}{}}
%\rfoot[\fancyplain{}{}]{\fancyplain{}{}}
% Citation format
\bibliographystyle{apalike}
\bibpunct{[}{]}{;}{a}{,}{,}

\begin{document}

%Title page (This is generate automatically from the commands above)
\begin{titlepage}
\noindent {\large \textbf{\thesisAuthor}}
\vspace{2cm}

\noindent {\Huge \thesisTitle}
\vspace{2cm}

\noindent \thesisType, \thesisDate 
\vspace{2cm}

\noindent Artificial Intelligence Group\\ Department of Computer and Information Science\\ Faculty of Information Technology, Mathematics and Electrical Engineering\\

\vfill
\begin{center}
\includegraphics[width=3cm]{ntnu.pdf}
\end{center}
\end{titlepage}

\thispagestyle{empty}

\cleardoublepage

\frontmatter

\section*{Abstract}


In this paper the different mechanisms that enable the evolution of altruistic behaviour in evolutionary computationare explored. 
The research is motivated by the need for autonomous evolving robots to evolve relevant traits needed to avoid extinction in a dynamic environment. 
The paper contains a structured litterature review on the field of altruism in artificial evolution that places this work in context with contemporary research. 

%\begin{itemize}
%\item the field of research
%\item a brief motivation for the work
%\item what the research topic is and
%\item the research approach(es) applied. 
%\item contributions
%\end{itemize}

%The abstract length should be roughly half a page of text --- without lists, tables or figures.  

\clearpage

\section*{Preface}



\vspace{1cm}

This paper was written as a specialisation project in the autumn of 2013 at the Norwegian University of Science and Technology. It was officially supervised by Pauline Haddow with assistance by Jean-Marc Montanier.
%The preface includes the facts - what type of project, where it is conducted, who supervised and any acknowledgements you wish to give. 

\vfill

\hfill \thesisAuthor

\hfill Trondheim, \today

\clearpage

\tableofcontents

%\listoffigures

%\listoftables

\mainmatter

\chapter{Introduction}
\label{cha:Introduction}

This paper explores the different mechanisms that are responsible for the evolution of altruistic behaviour and in the most extreme case, the act of self-sacrifice.
The intented use is within the field of sub-symbolic AI where wanted behaviour is often found by mimicking the mechanisms of natural evolution. 
The goal of the paper is to shed light on what mechanisms we know to exist from nature and how these have been utilized in the artificial evolution of altruism.
Section \ref{sec:BackgroundAndMotivation} gives a cursory introduction to the field of artificial evolution and outlines why research on altruism is of interest.
Chapter \ref{cha:STL} gives a review of the existing literature related to the mechanisms behind altruism and the simulation of these mechanisms in simulated 
environments. Chapter \ref{cha:discussion} Gives a discussion of some of the questions that arise from the literature review and proposes a research question
that adresses the most central problem. Some thoughts on how this question should be researched is also given. The protocol for the structured literature review and the documentation of the review is given in \ref{cha:appendices}

%All chapters should begin with an introduction before any sections begin. Further, each sections begins with an introduction before  subsections begin. Chapters with just one section or sections with just one sub-section, should be avoided. Think carefully about chapter and section titles as each title stand alone in the table of contents (without associated text) and should convey meaning for the contents of the chapter or section. 

%In all chapters and sections it is important to write clearly and concisely. Avoid repetitions and if needed, refer back to the original discussion or presentation. Each new section, subsection or paragraph should provide the reader with new information and be written in your own words. Avoid direct quotes. If you use direct quotes, unless the quote itself is very significant, you are conveying to the reader that you are unable to express this discussion or fact yourself. Such direct quotes also break the flow of the language (yours to someone else's).   




\section{Background and Motivation}\label{cit}
\label{sec:BackgroundAndMotivation}

This section presents the theoretical background for the literature review on the mechanisms behind altruism in \ref{cha:review}. The goal is to clearly show where this research belongs and where it comes from. This section also presents the motivation for studying altruism in evolutionary computation, giving a rationale on why this topic is a worthwhile pursuit in general.

\subsection{Theoretical Background}

In the field of sub-symbolic AI there is interest in evolving solutions to problems by mimicking the mehcanisms of natural evolution. A sub-field of this is concerned with
evolving altruistic behaviour.

In evolutionary computation the solution to a problem has a genetic representation which is analogous to the genes of an individual in biology and a phenotype which is the expressino of the genotpye. The distiction between the two can be seen as the distinction between the blueprints for a mechanical object and the mechanical object themselves. A group of phenotypes is called a population and the process of optimizing the solutions are done in a process mimicking natural evolution where the best individuals in the population are combined to create new solutions in a new generation. 


Self-Sacrifice is a form of extreme altruism and In order to understand the subject of the evolution of altruism in general and to explore the various conditions under which altruism is evolved I conducted a systematic literature review. 


\cite{montanier_environment-driven_2013} presents a partial review of the most recognized mechanisms thataccount for the emergence of altruism. In this review, the mechanisms are divided into four categories:
\begin{description}
\item[Kin-selection] {The individuals that benefit from the altruistic deed are closely related to the altrist and are also harbours this capability, thus ensuring the survival of the gene}
\item[Group-selection] {Groups are created randomly containing altruistic and egotistical individuals and the altruists help ensure the survival of the group as a whole. The groups are reorganized at random after some predefined amount of time has passed and without this the altruists would go extinct within their own group.}
\item[Tag-recognition]{Phenotypic traits are used to idenitfy similarities in the genome, which altruistic individuals use to identify each other to gain selective advantage}
\item[Environment-viscosity] {In viscous populations, there is a greater chance that the benefit os altruistic actions goes to closely related individuals. This can be seen as a mechanism that ensures kin-selectino}
\end{description}


\subsection{Motivation}

The main area of interest in altruism in evolutionary computation is within the field of embodied evolution where agents are continously adapting to the environment. Creating the mechanisms for the evolution of altruistc behaviour or self-sacrifice is beneficial when this increases the overall fitness of the population with regards to the task to be solved. One instance where the most acute form of altruism could be of use is a group of autonomous robots exploring a glacier where further exploration is only possible if one of the robots drives into a crevasse to form a bridge that the others can runover.    
\chapter{A review of the literature}
\label{cha:review}

In this chapter follows a review of the literature on the mechanisms behind the
evolution of altruism. \ref{sec:biology} gives an introduction to the theory behind the evolution of altruism from theoretical biology.  
\ref{sec:cs} Gives an overview of the research that has been done on evolving altruism in simulated enviornments using the mechanisms described in theoretical biology.

\section{Altruism in Theoretical Biology}
\label{sec:biology}
A general framework for classification of the theoretical models of the evolution of altruism and cooperation is suggested in \cite{lehmann_evolution_2006} where four main categories are presented:  

In the classic texts by Hamilton such as \cite{w._d._hamilton_evolution_1963}, \cite{hamilton_genetical_1964-1} and \cite{hamilton_genetical_1964} Hamilton outlines the notion
that phenotypic traits can be used to identify individuals with similar geneotypes. This idea is further explored and named the 'Green Beard' effect in \cite{dawkins_selfish_2006}. 
This idea is further extended in \cite{agrawal_kin_2001} by "defining the conditions for the evolution of kin-directed altruism  when recognizers are permitted to make acceptance (type I) and rejection (type II) errors in the identification of social partners with respect to kinship."

\section{Altruism in Computer Science}
\label{sec:cs}

\cite{floreano_evolution_2008} presents four different algorithms that may lead to altruistic cooperation. Both selection at the level of the individual and team selection is tested. The experiments simulate ants foraging for food items where two ants can bring back more food by cooperationg than the two separatly can by bringing a food item each. The associated cost is that each ant get less food in return than when cooperating. Higher levels of altruism was observed when using team-level selection and more homogeneous teams had higher overall fitness. 
\cite{martijn_brinkers_evolution_1999} did simulations of the evolution of altruism with kin-recognition where the altruistic act was indeed self-sacrifice. Agents were placed on a grid, and the grid had parts with land and parts with water. The goal was to forage for food, and agents could drive into the water forming a bridge between two pieces of land so that others could reach the food that existed on the other side. However, this was based on a very simple simulation where the genome evolved was the probability that an agent would drive straight ahead when there was water in front of it. An important point here is that although the results showed that the probability increased when closely related individuals could benefit from it, the act itself was not a behaviour emerging from necessity, rather than by design. 
Experiments on kin selection in viscous populations was done by \cite{dulk_evolution_2000} exploring the effect it has on the evolution of altruism.This concept is also explored from the viewpoint of theoretial biology in \cite{mitteldorf_population_2000} 
\cite{montanier_surviving_2011} uses an experimental setup where autonomous robotic agents must forage for food and there is a chance that the situation of the tragedy of commons might occur. The fitness function is implicit by having the robots exchange genomes with every other robot it meets during a generation. The robot then chooses a genome to use at random from its list of genomes and uses a slightly modified version of this. This is interesting because low viscosity increases the fitness at the population level. Altruism is still observed and to a certain degree tuned by introducing a mechanism for kin-selection. 









 
%Having a template to work from provides a starting point. However, for a given project, a slight variation in the template may be required due to the nature of the given project. Further, the order in which the various chapters and sections will be written will also vary from project to project but will seldom start at the abstract and sequentially follow the chapters of the report. One critical reason for this, is that you need to start writing as early as possible and you will begin to write up where you are currently focusing. However, do not leave the abstract until the end. The abstract is the first thing anyone reads of an article or thesis --- after the title; and thus it is important that it is very well written. Abstracts are hard to write so create revisions throughout the course of your project as your project progresses.  

%This introduction to background and motivation should state where this project is situated in the field and what the key driving forces motivating this research are. However, keep this section brief as it is still part of the introduction. The motivation will be further extended in chapter~\ref{T-B}, presenting your complete state-of-the-art. 

%Note that this template uses italics to highlight where latin wording is inserted to represent text and the text of the template that we wish to draw your attention to. The italics themself are not an indication that such sections should use italics.  

%\section{Goals and Research Questions}
%\label{sec:Goals and Research Questions}

%A masters is a research project and thus there needs to be a question(s) that need answered. Such questions are often a very important part of the results that come out of the specialisation project. For those following the one year masters project, it is desirable to create such questions as early as possible as   The formation of such questions provide both an important driving force for the masters project and provide clarity as to the goals sought. However, one will expect to refine the questions and thus the final path of the masters as work progresses. However any refinements should be conducted with care so as to avoid that the original aims, and previous work are not lost.  
%It is always good to have one (or max 2) key questions and perhaps some sub questions. 

%\begin{description}
%\item[Goal] {\it Evolve observable and quantifiable altruistic beahaviour (That solves some problem in a new way) } 
%\end{description}

%Your goal/objective should be described in a single sentence. In the text under you can expand on this sentence to clarify what is meant by the short goal description. 
%The goal of your work is what you are trying to achieve. This can either be the goal of your actual project or can be a broader goal that you have taken steps towards achieving. Such steps should be expressed in the research questions. 
%Note that the goal is seldom to build a system. A system is built to to enable experiments to be conducted. The research question/goal would be the goal that the system is implemented to meet.  


%\begin{description}
%\item[Research question 1] {\it A clever research question about strong altruism}
%\end{description}

%Some more explanation here

%Each research question provides a sub-goal and these should be precise and clearly stated enabling the reader to match your results to the original goals. They will also form the driving force for the experimental plan. 

%\begin{description}
%\item[Research question 2] {\it Lorem ipsum dolor sit amet, consectetur adipiscing elit.}
%\end{description}

%{\it Lorem ipsum dolor sit amet, consectetur adipiscing elit. Nam consequat pulvinar hendrerit. Praesent sit amet elementum ipsum. Praesent id suscipit est. Maecenas gravida pretium magna non }

%\section{Research Method}
%\label{sec:researchMethod}

%What methodology will you apply to address the goals: theoretic/analytic, model/abstraction or design/experiment? This section will describe the research methodology applied and the reason for this choice of research methodology.  

%\section{Contributions}
%\label{sec:IntroContributions}
%
%The main description of the contributions will come in chapter~\ref{cont} after the results are presented. This section just provides a brief summary of the main contributions of the work. This section can also be left out, leaving all discussions in chapter~\ref{cont}.
%
%The format of this section will generally follow the following format:
%{\it
%Donec non turpis nec neque egestas faucibus nec id neque. Etiam consectetur, odio vitae gravida tempus, diam velit sagittis turpis, a molestie ligula tellus at nunc. Nam convallis consequat vestibulum. Proin dolor neque, dapibus a pellentesque a, commodo a nibh.}

%\begin{enumerate}
%\item {\it Lorem ipsum dolor sit amet, consectetur adipiscing elit.}
%\item {\it Lorem ipsum dolor sit amet, consectetur adipiscing elit.}
%\item {\it Lorem ipsum dolor sit amet, consectetur adipiscing elit.}
%\end{enumerate}
%

%\section{Thesis Structure}
%\label{sec:thesisStructure}

%This section provides the reader with an overview of what is coming in the next chapters. You want to say more than what is explicit in the chapter name, if possible, but still keep the description short and to the point. 


%\chapter{Background Theory and Motivation}\label{T-B}
%\label{cha:TheoryAndBackground}

%{\it Lorem ipsum dolor sit amet, consectetur adipiscing elit. Nam consequat pulvinar hendrerit. Praesent sit amet elementum ipsum. Praesent id suscipit est. Maecenas gravida pretium magna non interdum. Donec augue felis, rhoncus quis laoreet sed, gravida nec nisi. Fusce iaculis fermentum elit in suscipit.}


%\section{Background Theory}
%\label{sec:no1}

%The background theory depth and breadth depends on the depth needed to understand your project in the different disciplines that your project crosses.  It is not a place to just write about everything you know that is vaguely connected to your project. The theory is here to help the reader that does not know the theoretical basis of your work so that he/she can gain sufficient understanding to understand your contributions. In particular, the theory section provides an opportunity to introduce terminology that can later be used without disturbing the text with a definition.  In some cases it will be more appropriate to have a separate section for different theory. However, watch that you don't end up with too short sections. Subsections may also be used to separate different background theory. 

%When introducing techniques or results, always reference the source. Be careful to reference the original contributor of a technique and not just someone who happens to use the technique. For relevant results to your work, you would want to look particularly at newer results so that you have referenced the most up-to-date work in your area. If you don't have the source handy when writing, mark the test that a reference is needed and add it later. 

%Web pages are not reliable sources --- they might be there one day and removed the next; and thus should be avoided, if possible. A verbal discussion is not a source and should not be referenced or described in the text.  

%The bulk of citations in the report will appear in section~\ref{cit}. However, you will often need to introduce some terminology and key citations already in this chapter. 

%You can cite a paper in the following manners: 

%\begin{itemize}
%\item when referring to authors:\\
 %\citet{authorson10:_secon_best_paper_in_world} stated something rather nice.
%\item to cite indirectly: \\
% Papers should be written nicely \citep{authorson10:_secon_best_paper_in_world}\\
%or\\
%In \cite{authorson10:_secon_best_paper_in_world}, a less detailed template was presented.
%\item To just cite the authors: \\
%\citeauthor{authorson10:_secon_best_paper_in_world} wrote a nice paper.
%\item Or just the year: \citeyear{authorson10:_secon_best_paper_in_world}.
%\item You can even cite specific pages: \citet[p. 3]{authorson10:_secon_best_paper_in_world}.
%\end{itemize}
%
%\vspace{0.5cm}
%
%\noindent
%{\bf Introducing figures:} \\

%\begin{figure}[ht]
%\begin{center}
%\includegraphics[width=0.5\columnwidth]{figs/figure1.pdf}
%\caption[Boxes and arrows are nice]{Boxes and arrows are nice (adapted from \citet{authorson10:_secon_best_paper_in_world})}
%\label{fig:BoxesAndArrowsAreNice}
%\end{center}
%\end{figure}

%Remember that when you borrow figures you should always credit the original author --- such as Figure \ref{fig:BoxesAndArrowsAreNice} (adapted from \citet{authorson10:_secon_best_paper_in_world}). Also don't just put the figure in and leave it to the author to try to understand what the figure is. The figure should be put in to convey a message and you need to help the author to understand the message intended by explaining the figure in the text. 
%
%\vspace{0.5cm}
%
%\noindent
%{\bf Introducing tables in the report: }\\

%\begin{table}[htdp]
%\begin{center}
%\begin{tabular}{|c|c|c|c|c|}\hline\hline
%This & is & a & nice & table\\\hline
%This & is & a & nice & table\\\hline\hline
%\end{tabular}
%\caption{Example Table}
%\end{center}
%\label{tab:ExampleTable}
%\end{table}%

%As you can see from Table \ref{tab:ExampleTable}, tables are nice. However, again, you need to discuss the contents of the table in the text. You don't need to describe every entry but draw the authors attention to what is important for he/she to glean from the table. 

\chapter{Discussion}
\label{cha:discussion}

Some introductory text here

\section{Evaluation}

Exploring the evolution of altruistic behaviour and in particular the relationship between the evolution of altruism and genetic diversity leads to the question of 
whether or not kin-recognition is a precondition for the evolution of self sacrifice. 


\begin{description}
\item[Research question] {\it Will self sacrifice be possible without having any form of recognition of
    kin between recipients and benefactors?}
\end{description}

In this context, self sacrifice is thought of as relinquishing ones own possibility of further dissemination of ones own genes in order to ensure the possibility of another's.
Recognition of kin implies that the benefactor has a way of discriminating between those who have a genetic composition that is close to its own and those who do not.


\chapter{Systematic Literature Review Protocol}\label{T-B}
\label{cha:STL}
%\chapter{Evolving Self-Sacrifice: A Systematic Literature Review}\label{T-B}
%\label{cha:STL}

%\section{Introduction}
%\label{sec:STLintro}


The systematic literature review was performed using the guidelines for systematic 
literature review in software engineering presented in \cite{keele_guidelines_2007}.
The review protocol is presented along with the documentation of each step.
The literature review process is divided into 8 steps: %Check Kitchenham 2007

\begin{description}
\item[Step 1] {\it Defining review questions}

\item[Step 2] {\it Defining the systematic literature review protocol}

\item[Step 3] {\it Search for relevant studies}

\item[Step 4] {\it Selection of studies}

\item[Step 5] {\it Quality assessment}

\item[Step 6] {\it Data Collection}

\item[Step 7] {\it Data synthesis and anaylysis}

\item[Step8] {\it Dissemination}

\end{description}

\clearpage 

\section{Defining the review questions}
The first step in the systematic review process was to formalize the the goal of the review into review questions that the review is meant to answer. The goal of the review was to answer the following questions:

\begin{description}
\item[RQ1] {\it What are the mechanisms that allow altruistic behaviour to evolve?} 
\item[RQ2] {\it What are the most important factors in determining the degree of altruism displayed?}
\item[RQ3] {\it What is the level of altruism that has been achieved in artificial evolution?}

\end{description}

%\subsection{Defining the Systematic review Protocol}

%I'm not sure what to write here as I'm detailing the protocol as I go along.

\section{Search for relevant studies}

To perform the search in a systematic way I compiled a list of of relevant sources which would be the subject to systematic query. I decided to use the list compiled in \cite{Lillegraven_design_2010} as a starting point as it presented a list of relevant sources both for research on computer science in general and Artificial Intelligence. 
%In Addition to the sources identified there I also added Google Scholar and Science Direct as suggested by 
%\cite{Brereton_lessons_2007}. Google Scholar would be my primary source in identifying relevant papers The list of sources is shown in table \ref{table:sources}
 


\begin{center}
    \begin{tabular}{|l|c|r|}
    \hline
    Source                  &   Type                        & URL \\ \hline \hline
    ACM Digital Library     &   Digital Library             & \url{http://portal.acm.org/dl.cfm} \\  \hline %Subscription
    IEEE Xplore             &   Digital Library             & \url{http://ieeexplore.ieee.org/} \\ \hline   %Subscription
    CiteSeerX               &   Digital Library             & \url{citeseerx.ist.psu.edu} \\ \hline         %Free
    Web of Knowledge        &   Digital Library             & \url{http://wokinfo.com/} \\ \hline           %Free
%    Google Scholar          &   Bibliographic Database      & \url{http://scholar.google.com} \\ \hline     %Free
    Journal of AI Resarch   &   Journal                     & \url{http://jair.org/} \\  \hline      %Free
    References in papers    &   N/A                         & N/A \\\hline
    \label{table:sources}
    \end{tabular}
\end{center}

\subsection{Searching the online resources }
Following the methodology in \cite{oates_researching_2005} I created groups of search terms that were synonyms or similar in meaning. The purpose of this was to exploit the possibility of using boolean search strings in modern digital libraries. The search for relevant literature is a continous processand I present here the tables of search terms I used over the course of the project. The table in \ref{table:terms1} shows the table I ended up using.% At This point I was still looking at altruism in swarm robotics, which I later abandoned to avoid this self-imposed constriction. Many of the papers I found were of use in general, despite this. 

%\begin{center}
%   \begin{tabular}{| l | c | c | r |}
%     \hline
%      & Group 1 & Group 2 & Group 3 \\ \hline
%      Term 1 & Altruism & Evolution & Swarm Robots \\ \hline
%      Term 2 & Tragedy of Commons & Natural Selection & Swarm Intelligence \\ \hline
%      Term 3 & Collaboration & Evolving & Swarm Behaviour \\ \hline
%      Term 4 & & Evolutionary & \\ \hline
%      Term 5 & & Genetic & \\
%      \hline
%       \label{table:terms1}
%       \end{tabular}
%end{center}
%     
%his provided me with a set of 
%
%begin{equation}
%label{eq:comb}
%*5*3 = 60
%end{equation}
%
\noindent
combinations of search terms. Combining the different search terms created the search string in \ref{eq:search}
\\
%\noindent
%begin{multline}
%label{eq:search}
%"Swarm Behaviour" \lor "Swarm Robotics" \lor "Swarm Intelligence") \land \\
%Evolutionary \lor Evolution \lor Evolving \lor Natural Selection \lor Genetic) \land \\
%Altruism \lor Altruistic \lor Tragedy of Commons) \\
%end{multline}
%
\begin{center}
    \begin{tabular}{| l | c | r |}
      \hline
       & Group 1 & Group 2 \\ \hline
       Term 1 & Altruism 	& Evolution  \\ \hline
       Term 2 & Self-Sacrifice	& Natural Selection \\ \hline
       Term 3 & 		& Evolving   \\ \hline
       Term 4 & 		& Evolutionary  \\ \hline
       \hline
	\label{table:terms2}
       \end{tabular}
      \end{center}
      
This provided me with a set of 
\begin{equation}
\label{eq:comb2}
2*4 = 8 
\end{equation}

%\noindent
%pairs of combinations of search terms that was combined as shown in \ref{eq:search2}


%\begin{center}
%    \begin{tabular}{| l | c | c | r |}
%      \hline
%       & Group 1 & Group 2 & Group 3 \\ \hline
%       Term 1 & Altruism & Evolution & Kin-selection\\ \hline
%       Term 2 & Altruistic & Natural Selection & Viscosity\\ \hline
%       Term 3 & Self-sacrifice & Evolving &  \\ \hline
  %     Term 4 & & Evolutionary & \\ \hline
 %      \hline
%	\label{table:terms1}
%	\end{tabular}
%\end{center}

%This provided me with a set of 

%\begin{equation}
%\label{eq:comb2}

%2*4*2 = 16 

%\end{equation}

combinations of search terms combined in the boolean expression:

\begin{center}
\begin{multline}
\label{eq:search2}
(Altruism \lor Altruistic ) \land \\
(Evolution \lor Natural Selection\ \lor Evolving \lor Evolutionary)
\end{multline}
 \end{center}
%\subsection{Searching offline}
%To limit the number of results, the search was set to only return results published
%in the last 5 years. The rationale was that this field of research is still in
%its infancy and that the most relevant studies would be the most recent. This turned 
%out to be a wrongful assumption, and later iterations did not limit the search
%based on year of publication.

%\subsubsection{ACM Digital Library}
For the ACM Digital Library, the number of results on the original search query
was so large that it had to be further limited by only including entries from 
relevant publications. Of the publications that returned matches for the query,
only two of them were included in the final search:

\begin{itemize}

%\item{Proceedings of the 13th annual conference companion on Genetic and evolutionary computation (255)}
%\item{Proceedings of the 11th International Conference on Autonomous Agents and Multiagent Systems - Volume 3 (175) }
%\item{The 10th International Conference on Autonomous Agents and Multiagent Systems - Volume 3 (171)}
\item{Proceedings of the 9th annual conference on Genetic and evolutionary computation}
\item{Proceedings of the fourteenth international conference on Genetic and evolutionary computation conference companion}
\item{Proceeding of the fifteenth annual conference companion on Genetic and evolutionary computation conference companion}
\item{Autonomous Agents and Multi-Agent Systems}
\item{Evolutionary Computation}
\item{Proceedings of the fourth international joint conference on Autonomous agents and multiagent systems}
\item{Proceedings of The 8th International Conference on Autonomous Agents and Multiagent Systems - Volume 2 }
\item{Artificial Life and Robotics}
\item{Proceedings of the 2004 international conference on Multi-Agent and Multi-Agent-Based Simulation }
\item{Proceedings of the Twenty-Second international joint conference on Artificial Intelligence - Volume Volume Two}
\item{Artificial Intelligence}
\item{Autonomous Robots}
\item{Neural Networks}
\item{Artificial Intelligence Review }
\end{itemize}

The search was further refined by adding the constraint that the title had to include

$Altruism \lor Altruistic$



Springer Link allows filtering on research field, so the search was limited to Artificial Intelligence.

On CiteSeer, the constraint on the terms "Viscosity" and "selective fitness" was relaxed.

The search string for IEEE Xplore was also limited to the publications


\begin{itemize}
\item{ Evolutionary Computation, IEEE Transactions on}
\item{Computational Intelligence in Robotics and Automation, 1997. CIRA'97., Proceedings., 1997 IEEE International Symposium on}
\item{Intelligent System and Knowledge Engineering, 2008. ISKE 2008. 3rd International Conference on}
\end{itemize}
The search on Web of Knowledge was refined to include only results from the research domains Science Technology and computer science
\subsubsection{Search Results}

Applying the search string in \ref{eq:search2} to the sources in \ref{table:sources} yielded the results shown in table \ref{table:SearchResults}

\begin{center}
    \begin{tabular}{|l|r|}
    \hline
	Source				& Hits 	\\ \hline
	Springer Link 			& 39 	\\ \hline
%    Google Scholar & 217 \\ \hline
	CiteSeer    			& 26 \\ \hline
    	ACM Digital Library 		& 25 \\ \hline
    	IEEE Xplore 			& 4  \\ \hline
	Web of Knowledge 		& 23 \\ \hline
    	Journal of AI Research 		& 0 	\\ \hline
	Other  				& 2 \\ 
    \hline
     \label{table:SearchResults}
   \end{tabular}
\end{center}

In addition to exploring the vast online resources I also searched available literature in the University Library and checked reference lists 
in the articles I read that were of particular interest if the theme they referenced fit some of my inclusion criteria or if the title alone fit one or more of my inclusion criteria.

%\begin{tabular}{| p{5cm} | p{5cm} |}
%	\hline
%	Title of paper found & Referenced in \\ \hline
%        The evolution of cooperation and altruism  a general framework and a classification of models & Evolution of Altruistic Robots \\ \hline
%\end{tabular}

	
\subsection{Selection of studies}

After applying the search strategy I began selecting the studies that were relevant for my research questions. To filter the number of studies found I employed a three stage screening process where the set of found articles were gradually culled according to a set of inclusion critereas. The three stage process was:

\begin{itemize}

\item screning based on title
\item Screening based on contents in the Abstract
\item Screening based on full-text reading
\item Screening based on quality

\end{itemize}

\subsubsection{Screening based on title}
The first level of screening was based on excluding articles based on the following criteria:

\begin{description}
\item[EQ1] {\it The main focus of the title is not within the field of computer science}
\item[EQ2] {\it It can be quickly determined from the title that the focus of the research is neither AI nor theoretical biology related to altruism}
\end{description}


\subsubsection{Abstract inclusion criteria screening}

The inclusion criterias that were used for the screening based on the contents in the abstract were:

\begin{description}
\item[IC1] {\it The paper focuses mainly on evolving altruistic behaviour using artificial evolution}
\item[IC2] {\it The paper focuses mainly on one of the mechanisms behind the evolution of altruistic behaviour in nature}
\end{description}

Before the full text inclusion criteria screening, the search results were as follows:
\begin{center}
    \begin{tabular}{|l|r|}
    \hline
	Source				& Hits 	\\ \hline
	Springer Link 			& 3 	\\ \hline
%    Google Scholar & 217 \\ \hline
	CiteSeer    			& 4 \\ \hline
    	ACM Digital Library 		& 5 \\ \hline
    	IEEE Xplore 			& 1  \\ \hline
	Web of Knowledge 		& 5 \\ \hline
    	Journal of AI Research 		& 0 	\\ \hline
	Other  				& 2 \\ \hline \hline
	Total				& 20 \\
    \hline
     \label{table:SearchResults}
   \end{tabular}
\end{center}


\subsubsection{Full text inclusion criteria screening}

\begin{description}
\item[IC4] {\it The paper focuses mainly on evolving altruistic behaviour using artificial evolution}
\item[IC6] {\it The paper recreates one or more of the settings in which altruistic behaviour evolve}
\item[IC7] {\it The paper studies the genetic preconditions for the evolution of altruistic behavior}
\end{description}

\subsubsection{Full text quality criteria screening}

\begin{description}
\item[QC1] {\it There is a clear statement of the aim of the research} 
\item[QC2] {\it The Study is put into context of other studies and research}
\end{description}

%\subsection{Quality assessment}
%The papers were screened based on the following quality criteria:

%\subsection{List of included papers}
%\noindent
%\begin{center}
%    \begin{tabular}{| p{5cm} | p{3cm} | r |}
%      \hline
%       Title & Author(s) & Source \\ \hline
%       	Evolution of Altruistic Robots & \begin{tabular}{l}Dario Floreano \\ Sara Mitri1\\ Andres Perez-Uribe \\ and Laurent Keller \\ \end{tabular} & Google Scholar  \\ \hline
%	The Evolution of Non-reciprocal Altruism & \begin{tabular}{l}Martijn Brinkers \\ Paul den Dulk\end{tabular} & ACM Digital Library \\ \hline
%	Evolution of Altruism in Viscous Populations: Effects of Altruism on the Evolution of Migrating Behavior & \begin{tabular}{l}Martijn Brinkers \\ Paul den Dulk\end{tabular} & ACM Digital Library \\ \hline
%	The Evolution of Altruistic Behavior & W. D. Hamilton & Google Scholar \\ \hline
	
 %      \end{tabular}
 %     \end{center}


\section{Data collection}
Given the exploratory nature of this literature review the data collection consisted of reading the material and
noting interesting points. 

\section{Data synthesis and analysis}

\section{Dissemination}
Dissemination means communicating the results, in this instance the review was handed in as part
of a project.

%Your motivation can be either application driven or technique/methodology driven. However in both cases, there will be an element of methodology driven due to the research focus of our group and the nature of a masters project.  
%What other research has been conducted in this area and how is it related to your work? The text should clearly illustrate why your goals and research questions are important to address. This section is thus where your literate review will be presented. It is important when presenting the review that you present an overview of the motivating elements of the work going on in your field and how these relate to your proposal, rather than a list of contributors and what they have done. This means that you need to extract the key important factors for your work and discuss how others have addressed each of these factors and what the advantages/disadvantages are with such approaches. As you mention other authors, you should reference their work. Note that the reference list reflects the literature you have read and have cited. This will only be a subset of the literature that you have read.


%\chapter{Evaluation and Conclusion}
%\label{cha:evaluationAndConclusion}

%{\it Lorem ipsum dolor sit amet, consectetur adipiscing elit. Nam consequat pulvinar hendrerit. Praesent sit amet elementum ipsum. Praesent id suscipit est. Maecenas gravida pretium magna non interdum. Donec augue felis, rhoncus quis laoreet sed, gravida nec nisi. Fusce iaculis fermentum elit in suscipit. }

%\section{Evaluation}
%\label{sec:Evaluation}

%When evaluating your results, avoid drawing grand conclusions, beyond that which your results can infact support. Further, although you may have designed your experiments to answer certain questions, the results may raise other questions in the eyes of the reader. It is important that you study the graphs/tables to look for unusual features/entries and discuss these aswell as discussing the main findings in the results. 

%\section{Discussion}
%\label{sec:Discussion}

%In the discussion it is important to include a discussion of not just the merits of the work conducted but also the limitations. 

%\section{Contributions}~\label{cont}
%\label{sec:Contributions}

%What are the main contributions made to the field and how significant are these contribution.  

%\section{Future Work}
%\label{sec:futureWork}

%Consider where you would like to extend this work. These extensions might either be continuing the ongoing direction or taking a side direction that became obvious during the work. Further, possible solutions to limitations in the work conducted, highlighted in ~\ref{sec:Discussion} may be presented. 
%\chapter{Structured Literature Review Protocol}
%\label{cha:STLP}

\backmatter

\addcontentsline{toc}{chapter}{Bibliography}
\bibliography{library}


%\chapter{Appendices}
%\label{cha:appendices}
%Here is the appendix
%Personal notes:

%TOOO
%Fix table of sources

\end{document}
